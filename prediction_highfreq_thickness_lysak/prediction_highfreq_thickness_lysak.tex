\documentclass{article}
\usepackage[letterpaper,margin=1in]{geometry}


\begin{document}
\begin{center}
\noindent \textbf{Prediction of high frequency gust response with airfoil thickness effects}\\
\end{center}
By: Peter D. Lysak, Dean E. Capone, Michael L. Johnson\\
Journal of Fluids and Structures\\
Notes by Dorien Villafranco\\

\noindent \textbf{1. Introduction} \\
One of the main sources of low frequency broadband noise is the unsteady blade loading that results from ingested turbulence. At higher frequencies where the relevant turbulent length scales are comparable to the airfoil thickness the flat plate approx. breaks down. Flat plate model does not account for distortion of the incident vortical gusts due to curved streamline near leading edge. Distortion results in attenuation of the high frequency components of the unsteady lift. \\

\noindent \textbf{2. Background}\\
To use panel methods successfully for the high frequency gust response, great care needs to be taken to ensure the panel resolution is fine enough to prevent oscillations when a vortex passes close to the surface.\\

\noindent \textbf{3. Theory}\\
Assume that the gust is a small amplitude vortical disturbance which is carried along by the mean flow. 



\end{document}